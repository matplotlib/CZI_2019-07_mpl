\documentclass[12pt,letterpaper]{article}  % min is 11pt

\usepackage[sort&compress,numbers]{natbib}
\usepackage[margin=1in,letterpaper]{geometry}  % min is .5 in
\usepackage[shortlabels]{enumitem}
\usepackage{graphicx}
\usepackage{caption}
\usepackage{footmisc}
\usepackage[hidelinks]{hyperref}
\usepackage{fancyhdr}
\usepackage{times}
\rhead{\emph{CZI: Essential Open Source Software for Science}}
\lhead{\emph{Matplotlib: Scientific Visualization}}
%\linespread{1.5}

\setlist[itemize]{noitemsep, topsep=0pt}

\pagestyle{fancy}

\begin{document}
\title{Matplotlib - Foundation of Scientific Visualization in Python}
\author{}
\maketitle

\section{Goals}

Matplotlib\cite{Hunter:2007} is the fundamental
data visualization library for the scientific Python Ecosystem, used
in conjunction with other foundational tools like NumPy and
SciPy \cite{Jones2001} by over a million\footnote{Estimated from \texttt{pypi}
and \texttt{conda} download numbers and unique monthly visitors to the documentation website} users.
Matplotlib is used across a wide spectrum of fields, including bio-medical imaging,
microscopy, and genomics \cite{Carpenter2006,Wolf2018,10.7717/peerj.453,
  Segata2011,10.1371/journal.pgen.1000695,HASHIMSHONY2012666,
  10.1093/bioinformatics/bts480,Carlile2014,Laganowsky2014,Jiangaac9462,
  10.3389/fninf.2014.00014}, and we expect this user base to grow as Python
  continues to be adopted in the life sciences. %%?citation needed that this still happening?
Matplotlib has been actively developed and maintained by a vibrant,
primarily volunteer, community over the last 16 years; however, given
the scale, scope, and importance of the project, we are at the limit
of what we can sustainably maintain and develop with mostly
volunteer effort.

This proposal asks for approximately 2.5 person-years of developer effort to support:

\begin{enumerate}[label=\alph*),noitemsep]
  \item fostering community, documentation, and tools to grow a rich ecosystem of domain-specific libraries built on Matplotlib;
  \item maintaining the library and contributor base;
  \item and developing a comprehensive plan to evolve the core architecture.
\end{enumerate}

\subsection{Fostering downstream projects}
\label{sec:downstream}
We will engage with libraries in the life sciences that are using
Matplotlib for their visualization to ensure that we are addressing
their problems and to collaborate on prototyping an end-to-end
solution, including data, plot representations, and a high-level user
API.  In particular we plan to engage with
\texttt{scikit-learn}\footnote{Also applying for Essential Open Source
Software for Science}, \texttt{CellProfiler}\footnote{Currently funded
by CZI\label{f:czi}}, \texttt{scanpy}\footref{f:czi},
\texttt{starfish}\footref{f:czi}, \texttt{nipy}, and
\texttt{scikit-image}\footref{f:czi}
\cite{10.7717/peerj.453,Carpenter2006, Wolf2018}.
Our goal is to make these libraries and others like them easier to
write.

The most common visualizations in a domain need to be fluid for
the end-practitioners, with the ``obvious'' customization options
exposed. Much of the domain-specific specialization is carried in the
structure, semantics and assumptions of the data, and in the standard
visualizations of the domain. These specializations can vary widely,
in contradictory ways, between domains. Because no high-level API can
simultaneously satisfy all of the visualization needs, there will
always be a need for domain-specific visualization libraries.

Through maintaining the code base and developing the architecture for
the next version, we will generate API guidelines so that downstream
libraries inter-operate well and ``feel'' similar. By building out
community, documentation, and tools, we will provide downstream
library developers resources for efficiently building their tools with
Matplotlib.

\subsection{Maintenance and Growth of Library and Community}

Matplotlib is a community-driven project, but we have grown to the
point where we need developers with the time to organize, plan, and
make decisions. Issues and PRs are submitted faster than our
volunteers can review them; for the past few years we have resolved
approximately 40 Issues and 30 PRs a month, but every month about 50
new Issues are opened and 40 PRs are submitted.  We have accumulated
about 1200 open Issues and 300 open PRs due to this imbalance. There
may be critical bug reports or insightful feature requests among the
former, while among the latter are useful contributions or bug fixes
that would improve Matplotlib for direct users and downstream
packages. The backlog is discouraging for new and occasional
contributors and distracting for core developers.

To maintain Matplotlib's health we need to:
%% these can probably be shortened further, to just stuff like Triage and on board
%%since explainations are in text
\begin{itemize}[noitemsep]
\item fix critical bugs and regressions;
\item triage the backlog of Issues and PRs in terms of topic, difficulty, and urgency and promptly triage newly opened Issues and PRs;
\item maintain backward compatibility and extensively document intentional changes;
\item on-board new contributors to sustain and diversify developer team;
\item and manage discussions about proposed enhancements, features, and API changes.
\end{itemize}

The requested support for developers is intended to complement and
facilitate, not replace, crucial volunteer work. We aim to better
co-ordinate and nurture their efforts, with the goal of growing and
sustaining a diverse community of volunteer and paid expert
contributors.  We view this proposal as the start of a new phase for
Matplotlib, in which we will need to find support from a variety of
sources in the coming years.

% Cite Ipython, Jupyter, Numpy as our models or inspiration in this?


\subsection{Road-map and Architecture}

Matplotlib needs sustained attention to update the library's
architecture for the next decade.  Developed over 15 years
ago\footnote{\url{https://www.aosabook.org/en/matplotlib.html}}
\cite{Hunter:2007}, it has served us well, but it does not reflect
recent developments in software design, data structures, and
visualization. We aim to take advantage of these developments to
improve the consistency, composability, and discoverabilty of the
API. The ultimate goal is facilitating rich interactive
domain-specific visualizations.


%% We will develop
%% plans for refactoring the core library to facilitate its extension and
%% to smooth interoperability among the core library and the various
%% domain-specific plotting tools.
%% We will investigate how to update Matplotlib's
%% internal data model to use the information embedded in structured
%% data and allow fluid updating.

\subsubsection{Homogenization of the Application Programming Interface (API)}
\label{sec:api_hom}
% should we merge this into the next section?
The library has grown organically over time through the contributions of over 900 individuals, accumulating many small inconsistencies in the API. For example, similar methods have different
argument order, e.g., \texttt{ax.text(x, y, s)} vs
\texttt{ax.annotation(s, (x, y))}, and some keyword (named) arguments can be
singular or plural, e.g., \texttt{color} vs \texttt{colors}.  These
subtle issues add friction for users, but are hard to fix without
breaking existing code for someone in our large user base.  Our goal is
to minimize breakage, while still unifying the API. Taking into account
\textbf{all} of the public interfaces, we will carefully consider which to leave as
they are, which to deprecate, and which to replace.

\subsubsection{API generalization}
\label{sec:api_gen}
Matplotlib has two main user-facing APIs, the \texttt{pyplot} API and
the \texttt{Object Oriented} (\texttt{OO}) API.  The former closely
follows MATLAB and is built around the concept of a global ``current
Axes'' drawing surface whereas the later requires explicit specification
of the \texttt{Axes} with each operation.

Although \texttt{pyplot} can be convenient for interactive terminal
usage, it is easy for the user to lose track of which Axes is the
``current Axes'', particularly when library functions, where code is
typically hidden from the user, are also using \texttt{pyplot}.  This
uncertainty frequently results in data being plotted to wrong and
surprising places.


The main namespace for plotting methods in the \texttt{OO} API is
the \texttt{Axes} class which avoids the ambiguity about which
\texttt{Axes} to use, however this
leads to three issues:
\begin{enumerate}
\item There is an API inconsistency between built-in plotting methods
  on the \texttt{Axes} class and third party domain-specific plotting
  functions, which are not \texttt{Axes} methods.
\item Some visualizations require plotting to multiple
  \texttt{Axes}es, which doesn't fit the model in which plots are made
  via \texttt{Axes} methods.
\item There are over 250 methods on the \texttt{Axes}, making it hard
  to use tab-completion to search for a method.
\end{enumerate}

To address these issues, we will move to a primary API in which
top-level functions take in data, style, and \texttt{Axes}(es) and
return \texttt{Artist}s.  These functions will use consistent naming
and call conventions, as discussed in \ref{sec:api_hom}, and return the
\texttt{Artist}s discussed in section \ref{sec:artists}. From the
large pool of plotting functions, augmented in collaboration with
downstream libraries, we will curate domain-specific namespaces to
facilitate discoverability.


\subsubsection{Smart Composite \texttt{Artist}s}
\label{sec:artists}
\texttt{Artist}s are the ``middle layer'' of Matplotlib that encodes a
plot's type, look, and data.  \texttt{Artist}s know how to translate
user specifications to rendered output, and users interact with the
\texttt{Artist} objects to update the data or aesthetics.

We need to embrace composite artists to provide a smarter, more uniform \texttt{Artist} API.
Matplotlib has a mix of ``primitive'' \texttt{Artist}s (e.g., lines, images, and text) and ``composite'' artists (e.g., the whole \texttt{Axes}). The mapping between the user API and the \texttt{Artist}s can be
one-to-one, but often one user call will generate many decoupled primitive \texttt{Artist}s.
For example, \texttt{hist} displays a histogram, notionally a single entity to the user,
but returns a list of independent \texttt{Rectangle} \texttt{Artist}s (one per bin).
If the user wants to update the data or the aesthetics, they must adjust each \texttt{Rectangle} independently.
Instead, the functions discussed in \ref{sec:api_gen} should return objects with
a uniform update interface, making interactive visualizations easier to develop.


\subsubsection{Data Model}
\label{sec:dm}
Matplotlib needs a consistent way for \texttt{Artist}s to access their
underlying data. Currently, each plotting method and \texttt{Artist}
handles sanitizing and storing data independently. Hence, some common
functionality, such as handling data with attached units (e.g.,
degrees Celsius, dates), is repeated throughout the code base. This
leads to inconsistencies across the library and makes it difficult for
users to write code that updates interactive explorations and
animations. Matplotlib also cannot exploit ``Structured
data''--combining multiple pieces of (possibly heterogeneous) data
with labels and metadata into single data structure; instead, users
must extract data elements and reassemble them as arguments to
Matplotlib plotting functions.

We will develop a data access API such that each \texttt{Artist} will have a single \texttt{Data} object responsible for holding the data. This model will be the technical underpinning to handle, exploit, and
update structured data in a coherent fashion. By removing direct data storage from the \texttt{Artist}s and defining an API for data sources we will enable

\begin{itemize}[noitemsep]
  \item native consumption of structured data,
  \item smart down-sampling of plotted data based on view limits,
  \item seamless updating of the underlying data, either interactively or via streams,
  \item and use of alternative data sources such as database queries or analytic functions.
\end{itemize}

This will decouple the implementation of the data sources from the
\texttt{Artist}s.  Downstream libraries will be able to provide
sophisticated data sources to the core library \texttt{Artists}, build
complex \texttt{Artist}s that use those sources, and create
specialized forms of the data model.

% \subsubsection{Additional Export Methods}
% % maybe nuke this section?
% Matplotlib \texttt{Figure}s can render to either raster or vector
% file formats, displaying the result in an interactive window or
% saving it to disk.  There is
% currently no good way to save and reopen a Matplotlib \texttt{Figure} or
% export it to another plotting library, such as \texttt{bokeh},
% \texttt{d3} or \texttt{QtCharts}.  In addition, due to the way Matplotlib
% internals are implemented, it is difficult to take advantage of GPUs to
% accelerate drawing. To address these problems, we will investigate adding two additional
% export paths.  One, at a high-level, will be suitable for subsequent re-input
% to Matplotlib and for interoperability with other high-level plotting
% libraries. The second, at a low scene-graph level, will be suitable for passing
% to a GPU.

\section{Expected outcomes, success evaluation and metrics}
\subsection{Maintenance}

Quantitatively evaluating maintenance work can be tricky---some Issues
or PRs take minutes to review while others can take days to months of
effort---but we believe that there is value at looking at the number
of open Issues and PRs.  We will reduce this number by closing Issues
and PRs faster than they are opened until a reasonable equilibrium is
reached. With the introduction of paid developers, NumPy has had
success in reversing the ever increasing trends in the number of open
Issues and Pull
Requests\footnote{https://github.com/seberg/numpy\_talk\_plots/blob/master/plots\_used\_in\_talk/issues\_prs\_delta.pdf}. Once
the backlog is handled, we aim to have all new Issues and Pull
Requests labeled within 7 days of being opened.

Improved contributor onboarding should yield an increase in the percentage of new contributor PRs that get resolved, a decrease in the amount of time it takes for resolution, and an increase the number of PRs submitted by non-core developers.


\subsection{Road-map and Architecture}

We will write a white paper with a road map documenting the proposed design, critical use-cases, and requirements for the data model and API overhaul.

% JMK Does this section need more?
% EF Would the prototyping just show what use of the API would look like?
%    How much of the actual mpl implementation would be expected?
%    What is actually feasible in a year?
%    It is OK for the proposal to be ambitious, but it should also
%    show some recognition of reality.

\section{Downstream Libraries}
To validate the design, we will develop end-to-end prototypes targeting one or more of the life-science libraries discussed in \ref{sec:downstream}, and qualitatively assess downstream library developers' engagement with these prototypes.

\section{Work Plan}

We request funding for the following:

\begin{itemize}[noitemsep]

\item Fund Thomas Caswell's position at 40\%.  Caswell is the Lead Developer of Matplotlib and an Associate
  Computational Scientist at Brookhaven National Laboratory.  His
  long-term experience, API design expertise, and project leadership
  are critical to the success of the work in this proposal.  He will work
  on all aspects of the proposal.
\item Hannah Aizenman's position for 12 months.  Aizenman has
  been a core developer of Matplotlib for three years. She has helped
  lead outreach efforts and is the author of a recent feature set.
  She is a PhD candidate in Computer Science, studying visualization at The
  Graduate Center, CUNY. Her work on the architecture of Matplotlib will be
  the basis of her PhD thesis. Aizenman will take
  the lead on the data model design and new-contributor on-boarding.
\item Summer support for Michael Grossberg, Aizenman's PhD advisor.
\item 12 months of a yet-to-be identified software engineer to support all aspects of the proposal but focusing on maintenance, prototyping, and engaging down-stream libraries.
\item Travel to key conferences (such as SciPy or PyCon) and for in-person meetings
\end{itemize}

We want to use this dedicated effort to leverage and empower the Matplotlib developer community.  In terms of direct work, equal amounts of time will be spent mentoring and reviewing code from community members and directly implementing features or fixing bugs.  All of the design work will be done in public with
input from the community. Part of this work is to develop the project Road Map.



\section{Existing Support}
Thomas Caswell has 4hrs/wk from Brookhaven National Lab to work on Matplotlib.


\clearpage
\bibliographystyle{ieeetr} % or named ?
\bibliography{biblo}

\end{document}
